\chapter{Introduction}

The continuous evolution of the OT and IoT worlds constantly creates new challenges in the cyber security field. One of the main challenges directly derives from the evolution of the modern networks. With the advent of the Internet of Things many networks, especially in the industrial production field, that were once small and isolated from external devices are now becoming bigger and bigger and connected to the external world. This change brings with it many advantages such as the possibility of performing remote monitoring and surveillance but also many risks in terms of the security of the network itself.

One of the main challenges in the surveillance and control of a network is the capability of performing the so-called device identification, namely knowing which devices are connected to it. In small sized networks this operation can be executed "by hand" by singularly registering each connected device. In big scales networks where each day tens, if not hundreds, of devices are connected and disconnected, e.g. the Wi-Fi network of a company were the employees may connect with tables, laptops and smartphones, this operation must be automated. 

Connected devices provide some identifiers, such as the MAC address, in order to allow the manager of a network to identify them. This identifiers however are not always trustworthy, they can in fact be falsified and/or modified, mostly by users with malicious intentions. Alternative methods should then be implemented to identify what a connected device really is without using any modifiable information.

The goal of this work is to develop a method to identify a device using Machine Learning techniques and, in particular, Neural Network classifiers. In particular we want to develop an architecture able to identify, once trained, devices with high accuracy. We also want this architecture to be as flexible as possible, namely we want it to be able to perform its task over many different type of networks. 